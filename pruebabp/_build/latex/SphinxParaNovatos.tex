% Generated by Sphinx.
\def\sphinxdocclass{report}
\documentclass[letterpaper,10pt,spanish]{sphinxmanual}
\usepackage[utf8]{inputenc}
\DeclareUnicodeCharacter{00A0}{\nobreakspace}
\usepackage{cmap}
\usepackage[T1]{fontenc}
\usepackage{amsfonts}
\usepackage{babel}
\usepackage{times}
\usepackage[Sonny]{fncychap}
\usepackage{longtable}
\usepackage{sphinx}
\usepackage{multirow}
\usepackage{eqparbox}


\addto\captionsspanish{\renewcommand{\figurename}{Figura }}
\addto\captionsspanish{\renewcommand{\tablename}{Tabla }}
\SetupFloatingEnvironment{literal-block}{name=Lista }



\title{Sphinx Para Novatos Guía de Aprendizaje}
\date{16 de September de 2016}
\release{1.0}
\author{Igor F. Dávalos Rojas}
\newcommand{\sphinxlogo}{}
\renewcommand{\releasename}{Publicación}
\setcounter{tocdepth}{1}
\makeindex

\makeatletter
\def\PYG@reset{\let\PYG@it=\relax \let\PYG@bf=\relax%
    \let\PYG@ul=\relax \let\PYG@tc=\relax%
    \let\PYG@bc=\relax \let\PYG@ff=\relax}
\def\PYG@tok#1{\csname PYG@tok@#1\endcsname}
\def\PYG@toks#1+{\ifx\relax#1\empty\else%
    \PYG@tok{#1}\expandafter\PYG@toks\fi}
\def\PYG@do#1{\PYG@bc{\PYG@tc{\PYG@ul{%
    \PYG@it{\PYG@bf{\PYG@ff{#1}}}}}}}
\def\PYG#1#2{\PYG@reset\PYG@toks#1+\relax+\PYG@do{#2}}

\expandafter\def\csname PYG@tok@gd\endcsname{\def\PYG@tc##1{\textcolor[rgb]{0.63,0.00,0.00}{##1}}}
\expandafter\def\csname PYG@tok@gu\endcsname{\let\PYG@bf=\textbf\def\PYG@tc##1{\textcolor[rgb]{0.50,0.00,0.50}{##1}}}
\expandafter\def\csname PYG@tok@gt\endcsname{\def\PYG@tc##1{\textcolor[rgb]{0.00,0.27,0.87}{##1}}}
\expandafter\def\csname PYG@tok@gs\endcsname{\let\PYG@bf=\textbf}
\expandafter\def\csname PYG@tok@gr\endcsname{\def\PYG@tc##1{\textcolor[rgb]{1.00,0.00,0.00}{##1}}}
\expandafter\def\csname PYG@tok@cm\endcsname{\let\PYG@it=\textit\def\PYG@tc##1{\textcolor[rgb]{0.25,0.50,0.56}{##1}}}
\expandafter\def\csname PYG@tok@vg\endcsname{\def\PYG@tc##1{\textcolor[rgb]{0.73,0.38,0.84}{##1}}}
\expandafter\def\csname PYG@tok@vi\endcsname{\def\PYG@tc##1{\textcolor[rgb]{0.73,0.38,0.84}{##1}}}
\expandafter\def\csname PYG@tok@mh\endcsname{\def\PYG@tc##1{\textcolor[rgb]{0.13,0.50,0.31}{##1}}}
\expandafter\def\csname PYG@tok@cs\endcsname{\def\PYG@tc##1{\textcolor[rgb]{0.25,0.50,0.56}{##1}}\def\PYG@bc##1{\setlength{\fboxsep}{0pt}\colorbox[rgb]{1.00,0.94,0.94}{\strut ##1}}}
\expandafter\def\csname PYG@tok@ge\endcsname{\let\PYG@it=\textit}
\expandafter\def\csname PYG@tok@vc\endcsname{\def\PYG@tc##1{\textcolor[rgb]{0.73,0.38,0.84}{##1}}}
\expandafter\def\csname PYG@tok@il\endcsname{\def\PYG@tc##1{\textcolor[rgb]{0.13,0.50,0.31}{##1}}}
\expandafter\def\csname PYG@tok@go\endcsname{\def\PYG@tc##1{\textcolor[rgb]{0.20,0.20,0.20}{##1}}}
\expandafter\def\csname PYG@tok@cp\endcsname{\def\PYG@tc##1{\textcolor[rgb]{0.00,0.44,0.13}{##1}}}
\expandafter\def\csname PYG@tok@gi\endcsname{\def\PYG@tc##1{\textcolor[rgb]{0.00,0.63,0.00}{##1}}}
\expandafter\def\csname PYG@tok@gh\endcsname{\let\PYG@bf=\textbf\def\PYG@tc##1{\textcolor[rgb]{0.00,0.00,0.50}{##1}}}
\expandafter\def\csname PYG@tok@ni\endcsname{\let\PYG@bf=\textbf\def\PYG@tc##1{\textcolor[rgb]{0.84,0.33,0.22}{##1}}}
\expandafter\def\csname PYG@tok@nl\endcsname{\let\PYG@bf=\textbf\def\PYG@tc##1{\textcolor[rgb]{0.00,0.13,0.44}{##1}}}
\expandafter\def\csname PYG@tok@nn\endcsname{\let\PYG@bf=\textbf\def\PYG@tc##1{\textcolor[rgb]{0.05,0.52,0.71}{##1}}}
\expandafter\def\csname PYG@tok@no\endcsname{\def\PYG@tc##1{\textcolor[rgb]{0.38,0.68,0.84}{##1}}}
\expandafter\def\csname PYG@tok@na\endcsname{\def\PYG@tc##1{\textcolor[rgb]{0.25,0.44,0.63}{##1}}}
\expandafter\def\csname PYG@tok@nb\endcsname{\def\PYG@tc##1{\textcolor[rgb]{0.00,0.44,0.13}{##1}}}
\expandafter\def\csname PYG@tok@nc\endcsname{\let\PYG@bf=\textbf\def\PYG@tc##1{\textcolor[rgb]{0.05,0.52,0.71}{##1}}}
\expandafter\def\csname PYG@tok@nd\endcsname{\let\PYG@bf=\textbf\def\PYG@tc##1{\textcolor[rgb]{0.33,0.33,0.33}{##1}}}
\expandafter\def\csname PYG@tok@ne\endcsname{\def\PYG@tc##1{\textcolor[rgb]{0.00,0.44,0.13}{##1}}}
\expandafter\def\csname PYG@tok@nf\endcsname{\def\PYG@tc##1{\textcolor[rgb]{0.02,0.16,0.49}{##1}}}
\expandafter\def\csname PYG@tok@si\endcsname{\let\PYG@it=\textit\def\PYG@tc##1{\textcolor[rgb]{0.44,0.63,0.82}{##1}}}
\expandafter\def\csname PYG@tok@s2\endcsname{\def\PYG@tc##1{\textcolor[rgb]{0.25,0.44,0.63}{##1}}}
\expandafter\def\csname PYG@tok@nt\endcsname{\let\PYG@bf=\textbf\def\PYG@tc##1{\textcolor[rgb]{0.02,0.16,0.45}{##1}}}
\expandafter\def\csname PYG@tok@nv\endcsname{\def\PYG@tc##1{\textcolor[rgb]{0.73,0.38,0.84}{##1}}}
\expandafter\def\csname PYG@tok@s1\endcsname{\def\PYG@tc##1{\textcolor[rgb]{0.25,0.44,0.63}{##1}}}
\expandafter\def\csname PYG@tok@ch\endcsname{\let\PYG@it=\textit\def\PYG@tc##1{\textcolor[rgb]{0.25,0.50,0.56}{##1}}}
\expandafter\def\csname PYG@tok@m\endcsname{\def\PYG@tc##1{\textcolor[rgb]{0.13,0.50,0.31}{##1}}}
\expandafter\def\csname PYG@tok@gp\endcsname{\let\PYG@bf=\textbf\def\PYG@tc##1{\textcolor[rgb]{0.78,0.36,0.04}{##1}}}
\expandafter\def\csname PYG@tok@sh\endcsname{\def\PYG@tc##1{\textcolor[rgb]{0.25,0.44,0.63}{##1}}}
\expandafter\def\csname PYG@tok@ow\endcsname{\let\PYG@bf=\textbf\def\PYG@tc##1{\textcolor[rgb]{0.00,0.44,0.13}{##1}}}
\expandafter\def\csname PYG@tok@sx\endcsname{\def\PYG@tc##1{\textcolor[rgb]{0.78,0.36,0.04}{##1}}}
\expandafter\def\csname PYG@tok@bp\endcsname{\def\PYG@tc##1{\textcolor[rgb]{0.00,0.44,0.13}{##1}}}
\expandafter\def\csname PYG@tok@c1\endcsname{\let\PYG@it=\textit\def\PYG@tc##1{\textcolor[rgb]{0.25,0.50,0.56}{##1}}}
\expandafter\def\csname PYG@tok@o\endcsname{\def\PYG@tc##1{\textcolor[rgb]{0.40,0.40,0.40}{##1}}}
\expandafter\def\csname PYG@tok@kc\endcsname{\let\PYG@bf=\textbf\def\PYG@tc##1{\textcolor[rgb]{0.00,0.44,0.13}{##1}}}
\expandafter\def\csname PYG@tok@c\endcsname{\let\PYG@it=\textit\def\PYG@tc##1{\textcolor[rgb]{0.25,0.50,0.56}{##1}}}
\expandafter\def\csname PYG@tok@mf\endcsname{\def\PYG@tc##1{\textcolor[rgb]{0.13,0.50,0.31}{##1}}}
\expandafter\def\csname PYG@tok@err\endcsname{\def\PYG@bc##1{\setlength{\fboxsep}{0pt}\fcolorbox[rgb]{1.00,0.00,0.00}{1,1,1}{\strut ##1}}}
\expandafter\def\csname PYG@tok@mb\endcsname{\def\PYG@tc##1{\textcolor[rgb]{0.13,0.50,0.31}{##1}}}
\expandafter\def\csname PYG@tok@ss\endcsname{\def\PYG@tc##1{\textcolor[rgb]{0.32,0.47,0.09}{##1}}}
\expandafter\def\csname PYG@tok@sr\endcsname{\def\PYG@tc##1{\textcolor[rgb]{0.14,0.33,0.53}{##1}}}
\expandafter\def\csname PYG@tok@mo\endcsname{\def\PYG@tc##1{\textcolor[rgb]{0.13,0.50,0.31}{##1}}}
\expandafter\def\csname PYG@tok@kd\endcsname{\let\PYG@bf=\textbf\def\PYG@tc##1{\textcolor[rgb]{0.00,0.44,0.13}{##1}}}
\expandafter\def\csname PYG@tok@mi\endcsname{\def\PYG@tc##1{\textcolor[rgb]{0.13,0.50,0.31}{##1}}}
\expandafter\def\csname PYG@tok@kn\endcsname{\let\PYG@bf=\textbf\def\PYG@tc##1{\textcolor[rgb]{0.00,0.44,0.13}{##1}}}
\expandafter\def\csname PYG@tok@cpf\endcsname{\let\PYG@it=\textit\def\PYG@tc##1{\textcolor[rgb]{0.25,0.50,0.56}{##1}}}
\expandafter\def\csname PYG@tok@kr\endcsname{\let\PYG@bf=\textbf\def\PYG@tc##1{\textcolor[rgb]{0.00,0.44,0.13}{##1}}}
\expandafter\def\csname PYG@tok@s\endcsname{\def\PYG@tc##1{\textcolor[rgb]{0.25,0.44,0.63}{##1}}}
\expandafter\def\csname PYG@tok@kp\endcsname{\def\PYG@tc##1{\textcolor[rgb]{0.00,0.44,0.13}{##1}}}
\expandafter\def\csname PYG@tok@w\endcsname{\def\PYG@tc##1{\textcolor[rgb]{0.73,0.73,0.73}{##1}}}
\expandafter\def\csname PYG@tok@kt\endcsname{\def\PYG@tc##1{\textcolor[rgb]{0.56,0.13,0.00}{##1}}}
\expandafter\def\csname PYG@tok@sc\endcsname{\def\PYG@tc##1{\textcolor[rgb]{0.25,0.44,0.63}{##1}}}
\expandafter\def\csname PYG@tok@sb\endcsname{\def\PYG@tc##1{\textcolor[rgb]{0.25,0.44,0.63}{##1}}}
\expandafter\def\csname PYG@tok@k\endcsname{\let\PYG@bf=\textbf\def\PYG@tc##1{\textcolor[rgb]{0.00,0.44,0.13}{##1}}}
\expandafter\def\csname PYG@tok@se\endcsname{\let\PYG@bf=\textbf\def\PYG@tc##1{\textcolor[rgb]{0.25,0.44,0.63}{##1}}}
\expandafter\def\csname PYG@tok@sd\endcsname{\let\PYG@it=\textit\def\PYG@tc##1{\textcolor[rgb]{0.25,0.44,0.63}{##1}}}

\def\PYGZbs{\char`\\}
\def\PYGZus{\char`\_}
\def\PYGZob{\char`\{}
\def\PYGZcb{\char`\}}
\def\PYGZca{\char`\^}
\def\PYGZam{\char`\&}
\def\PYGZlt{\char`\<}
\def\PYGZgt{\char`\>}
\def\PYGZsh{\char`\#}
\def\PYGZpc{\char`\%}
\def\PYGZdl{\char`\$}
\def\PYGZhy{\char`\-}
\def\PYGZsq{\char`\'}
\def\PYGZdq{\char`\"}
\def\PYGZti{\char`\~}
% for compatibility with earlier versions
\def\PYGZat{@}
\def\PYGZlb{[}
\def\PYGZrb{]}
\makeatother

\renewcommand\PYGZsq{\textquotesingle}

\begin{document}
\shorthandoff{"}
\maketitle
\tableofcontents
\phantomsection\label{index::doc}



\part{EL COMIENZO DEL CAOS}
\label{parte01:welcome-to-sphinx-para-novatos-s-documentation}\label{parte01:primera-parte}\label{parte01:el-comienzo-del-caos}\label{parte01::doc}\setbox0\vbox{
\begin{minipage}{0.95\linewidth}
\textbf{Es mi tabla de contenidos}

\medskip

\begin{itemize}
\item {} 
\phantomsection\label{parte01:id8}{\hyperref[parte01:el\string-comienzo\string-del\string-caos]{\emph{EL COMIENZO DEL CAOS}}} (\autopageref*{parte01:el-comienzo-del-caos})
\begin{itemize}
\item {} 
\phantomsection\label{parte01:id9}{\hyperref[parte01:primeros\string-pasos]{\emph{PRIMEROS PASOS}}} (\autopageref*{parte01:primeros-pasos})
\begin{itemize}
\item {} 
\phantomsection\label{parte01:id10}{\hyperref[parte01:buscando\string-ayuda\string-sobre\string-sphinx]{\emph{Buscando Ayuda sobre Sphinx}}} (\autopageref*{parte01:buscando-ayuda-sobre-sphinx})

\item {} 
\phantomsection\label{parte01:id11}{\hyperref[parte01:ayuda\string-encontrada]{\emph{Ayuda  Encontrada}}} (\autopageref*{parte01:ayuda-encontrada})

\item {} 
\phantomsection\label{parte01:id12}{\hyperref[parte01:requerimientos\string-iniciales]{\emph{Requerimientos Iniciales}}} (\autopageref*{parte01:requerimientos-iniciales})

\end{itemize}

\end{itemize}

\end{itemize}
\end{minipage}}
\begin{center}\setlength{\fboxsep}{5pt}\shadowbox{\box0}\end{center}


\chapter{PRIMEROS PASOS}
\label{parte01:primeros-pasos}\label{parte01:id3}
La instalación de \textbf{Sphinx} \footnote[1]{
Página web de \href{http://sphinx-doc.org/}{Sphinx} (http://sphinx-doc.org/)
} es sencilla. El uso del lenguaje \textbf{reStructuredText} es muy fácil de entender y sencillo de implementar cuando se lo utiliza a diario.
Sabiendo esto, puedes elaborar documentos que van desde una estructuración sencilla a una muy compleja.
Inicié esta referencia en español como una práctica más para entender como se puede sacar el máximo provecho a Sphinx y al lenguaje RST sin ser un desarrollador
ni tener conocimientos sobre \textbf{Python} \footnote[2]{
Página web de \href{http://www.python.org/}{Python} (http://www.python.org/)
} o similares.


\section{Buscando Ayuda sobre Sphinx}
\label{parte01:buscando-ayuda}\label{parte01:buscando-ayuda-sobre-sphinx}
La idea del creador de \code{Sphinx} (desarrollado en \code{Python}) fue tener a disposición de los desarrolladores una aplicación que les permita elaborar sus guías, manuales, y demás documentos de forma sencilla (para ellos, claro), dándoles las herramientas necesarias con las cuales puedan enlazar imágenes, tablas, código de sus programas y lo que vieran conveniente.

En este contexto, es que me topé con que la mayor parte de la documentación está más destinada a ser entendida por desarrolladores y no así por usuarios novatos o también llamados usuarios finales o comunes. Pues para un usuario común que necesita una aplicación para crear documentos de uso diario existen otras alternativas que son \emph{fáciles} de aprender, como por ejemplo: ApacheOpenOffice writer, LibreOffice writer, Abiword, MS Word, Wordperfect, Etc.


\section{Ayuda  Encontrada}
\label{parte01:ayuda-encontrada}\label{parte01:id6}
Existe bastante ayuda para aprender sobre reStructuredText, principalmente en su \href{http://docutils.sourceforge.net/rst.html}{página oficial} (http://docutils.sourceforge.net/rst.html) y en otras páginas encontrarás muchas referencias como en estas páginas que indico en {\hyperref[anexo:enlaces\string-utiles]{\emph{Enlaces Utiles}}} (\autopageref*{anexo:enlaces-utiles}).

Como indiqué en {\hyperref[parte01:primeros\string-pasos]{\emph{PRIMEROS PASOS}}} (\autopageref*{parte01:primeros-pasos}), la explicación sobre \code{Sphinx} es bastante técnica, osea se asume que el lector tiene conocimientos sobre \code{Python} o de programación, ya que las explicaciones están destinadas más que todo a la estructura del documento y no así a su maquetación final. Dejando así un gran vacío al usuario común sobre \emph{cómo} darle cuerpo al documento final. Si tienes conocimientos sobre \code{Tex} o \code{Latex} o similares, no tendrás ningún problema en adaptar la presentación final a tus gustos y necesidades; caso contrario quedas invitado a seguir leyendo esta \emph{Guía de Aprendizaje} y a realizar en tu editor de textos favorito, los ejemplos que  que se van desarrollando.


\section{Requerimientos Iniciales}
\label{parte01:id7}\label{parte01:requerimientos-iniciales}
Lo principal que necesitas es \emph{Saber leer y escribir}, eso es, saber leer exactamente lo que está escrito y escribirlo de igual modo. Esto referido a la sintaxis del formato, no al contenido que estarás creando. Por ejemplo, no obtienes lo mismo con \textbf{`esto'} = `esto'; que con \textbf{{}`esto{}`} = \emph{esto}.

necesitarás usar un \code{editor de textos} como el que viene instalado por defecto en tu sistema operativo.
\begin{itemize}
\item {} 
En \textbf{Linux} puedes disponer de los siguientes:
\begin{itemize}
\item {} 
medit, geany, gedit, nano y muchos más.

\end{itemize}

\item {} 
Para \textbf{Windows} tienes:
\begin{itemize}
\item {} 
block de notas, el notepad++ este último muy recomendado pues posee complementos para el resaltado de código y permite cambiar la codificación de caracteres.

\end{itemize}

\end{itemize}


\part{ANEXO}
\label{anexo:anexo}\label{anexo::doc}\label{anexo:id1}

\chapter{Enlaces Utiles}
\label{anexo:enlaces-utiles}\label{anexo:id2}
Listado de enlaces de utilidad para aprender sobre reStructuredText y Sphinx.

aaa


\part{Indices and tables}
\label{index:indices-and-tables}\begin{itemize}
\item {} 
\DUspan{xref,std,std-ref}{genindex}

\item {} 
\DUspan{xref,std,std-ref}{modindex}

\item {} 
\DUspan{xref,std,std-ref}{search}

\end{itemize}

\begin{thebibliography}{CIT2016}
\bibitem[CIT2016]{CIT2016}{\phantomsection\label{parte01:cit2016} 
Ejemplo de una cta bibliograf
(Que a veces es bibliográfica otras solo de referencia.)
}
\end{thebibliography}



\renewcommand{\indexname}{Índice}
\printindex
\end{document}
