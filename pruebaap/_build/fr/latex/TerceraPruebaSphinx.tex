% Generated by Sphinx.
\def\sphinxdocclass{report}
\documentclass[letterpaper,10pt,french]{sphinxmanual}
\usepackage[utf8]{inputenc}
\DeclareUnicodeCharacter{00A0}{\nobreakspace}
\usepackage{cmap}
\usepackage[T1]{fontenc}
\usepackage{amsfonts}
\usepackage{babel}
\usepackage{times}
\usepackage[Sonny]{fncychap}
\usepackage{longtable}
\usepackage{sphinx}
\usepackage{multirow}
\usepackage{eqparbox}


\addto\captionsfrench{\renewcommand{\figurename}{Fig. }}
\addto\captionsfrench{\renewcommand{\tablename}{Tableau }}
\SetupFloatingEnvironment{literal-block}{name=Code source }



\title{Tercera Prueba Sphinx Documentation}
\date{16 September 2016}
\release{1.0}
\author{Igor F. Dávalos Rojas}
\newcommand{\sphinxlogo}{\includegraphics{logo.png}\par}
\renewcommand{\releasename}{Version}
\setcounter{tocdepth}{1}
\makeindex

\makeatletter
\def\PYG@reset{\let\PYG@it=\relax \let\PYG@bf=\relax%
    \let\PYG@ul=\relax \let\PYG@tc=\relax%
    \let\PYG@bc=\relax \let\PYG@ff=\relax}
\def\PYG@tok#1{\csname PYG@tok@#1\endcsname}
\def\PYG@toks#1+{\ifx\relax#1\empty\else%
    \PYG@tok{#1}\expandafter\PYG@toks\fi}
\def\PYG@do#1{\PYG@bc{\PYG@tc{\PYG@ul{%
    \PYG@it{\PYG@bf{\PYG@ff{#1}}}}}}}
\def\PYG#1#2{\PYG@reset\PYG@toks#1+\relax+\PYG@do{#2}}

\expandafter\def\csname PYG@tok@gd\endcsname{\def\PYG@tc##1{\textcolor[rgb]{0.63,0.00,0.00}{##1}}}
\expandafter\def\csname PYG@tok@gu\endcsname{\let\PYG@bf=\textbf\def\PYG@tc##1{\textcolor[rgb]{0.50,0.00,0.50}{##1}}}
\expandafter\def\csname PYG@tok@gt\endcsname{\def\PYG@tc##1{\textcolor[rgb]{0.00,0.27,0.87}{##1}}}
\expandafter\def\csname PYG@tok@gs\endcsname{\let\PYG@bf=\textbf}
\expandafter\def\csname PYG@tok@gr\endcsname{\def\PYG@tc##1{\textcolor[rgb]{1.00,0.00,0.00}{##1}}}
\expandafter\def\csname PYG@tok@cm\endcsname{\let\PYG@it=\textit\def\PYG@tc##1{\textcolor[rgb]{0.25,0.50,0.56}{##1}}}
\expandafter\def\csname PYG@tok@vg\endcsname{\def\PYG@tc##1{\textcolor[rgb]{0.73,0.38,0.84}{##1}}}
\expandafter\def\csname PYG@tok@vi\endcsname{\def\PYG@tc##1{\textcolor[rgb]{0.73,0.38,0.84}{##1}}}
\expandafter\def\csname PYG@tok@mh\endcsname{\def\PYG@tc##1{\textcolor[rgb]{0.13,0.50,0.31}{##1}}}
\expandafter\def\csname PYG@tok@cs\endcsname{\def\PYG@tc##1{\textcolor[rgb]{0.25,0.50,0.56}{##1}}\def\PYG@bc##1{\setlength{\fboxsep}{0pt}\colorbox[rgb]{1.00,0.94,0.94}{\strut ##1}}}
\expandafter\def\csname PYG@tok@ge\endcsname{\let\PYG@it=\textit}
\expandafter\def\csname PYG@tok@vc\endcsname{\def\PYG@tc##1{\textcolor[rgb]{0.73,0.38,0.84}{##1}}}
\expandafter\def\csname PYG@tok@il\endcsname{\def\PYG@tc##1{\textcolor[rgb]{0.13,0.50,0.31}{##1}}}
\expandafter\def\csname PYG@tok@go\endcsname{\def\PYG@tc##1{\textcolor[rgb]{0.20,0.20,0.20}{##1}}}
\expandafter\def\csname PYG@tok@cp\endcsname{\def\PYG@tc##1{\textcolor[rgb]{0.00,0.44,0.13}{##1}}}
\expandafter\def\csname PYG@tok@gi\endcsname{\def\PYG@tc##1{\textcolor[rgb]{0.00,0.63,0.00}{##1}}}
\expandafter\def\csname PYG@tok@gh\endcsname{\let\PYG@bf=\textbf\def\PYG@tc##1{\textcolor[rgb]{0.00,0.00,0.50}{##1}}}
\expandafter\def\csname PYG@tok@ni\endcsname{\let\PYG@bf=\textbf\def\PYG@tc##1{\textcolor[rgb]{0.84,0.33,0.22}{##1}}}
\expandafter\def\csname PYG@tok@nl\endcsname{\let\PYG@bf=\textbf\def\PYG@tc##1{\textcolor[rgb]{0.00,0.13,0.44}{##1}}}
\expandafter\def\csname PYG@tok@nn\endcsname{\let\PYG@bf=\textbf\def\PYG@tc##1{\textcolor[rgb]{0.05,0.52,0.71}{##1}}}
\expandafter\def\csname PYG@tok@no\endcsname{\def\PYG@tc##1{\textcolor[rgb]{0.38,0.68,0.84}{##1}}}
\expandafter\def\csname PYG@tok@na\endcsname{\def\PYG@tc##1{\textcolor[rgb]{0.25,0.44,0.63}{##1}}}
\expandafter\def\csname PYG@tok@nb\endcsname{\def\PYG@tc##1{\textcolor[rgb]{0.00,0.44,0.13}{##1}}}
\expandafter\def\csname PYG@tok@nc\endcsname{\let\PYG@bf=\textbf\def\PYG@tc##1{\textcolor[rgb]{0.05,0.52,0.71}{##1}}}
\expandafter\def\csname PYG@tok@nd\endcsname{\let\PYG@bf=\textbf\def\PYG@tc##1{\textcolor[rgb]{0.33,0.33,0.33}{##1}}}
\expandafter\def\csname PYG@tok@ne\endcsname{\def\PYG@tc##1{\textcolor[rgb]{0.00,0.44,0.13}{##1}}}
\expandafter\def\csname PYG@tok@nf\endcsname{\def\PYG@tc##1{\textcolor[rgb]{0.02,0.16,0.49}{##1}}}
\expandafter\def\csname PYG@tok@si\endcsname{\let\PYG@it=\textit\def\PYG@tc##1{\textcolor[rgb]{0.44,0.63,0.82}{##1}}}
\expandafter\def\csname PYG@tok@s2\endcsname{\def\PYG@tc##1{\textcolor[rgb]{0.25,0.44,0.63}{##1}}}
\expandafter\def\csname PYG@tok@nt\endcsname{\let\PYG@bf=\textbf\def\PYG@tc##1{\textcolor[rgb]{0.02,0.16,0.45}{##1}}}
\expandafter\def\csname PYG@tok@nv\endcsname{\def\PYG@tc##1{\textcolor[rgb]{0.73,0.38,0.84}{##1}}}
\expandafter\def\csname PYG@tok@s1\endcsname{\def\PYG@tc##1{\textcolor[rgb]{0.25,0.44,0.63}{##1}}}
\expandafter\def\csname PYG@tok@ch\endcsname{\let\PYG@it=\textit\def\PYG@tc##1{\textcolor[rgb]{0.25,0.50,0.56}{##1}}}
\expandafter\def\csname PYG@tok@m\endcsname{\def\PYG@tc##1{\textcolor[rgb]{0.13,0.50,0.31}{##1}}}
\expandafter\def\csname PYG@tok@gp\endcsname{\let\PYG@bf=\textbf\def\PYG@tc##1{\textcolor[rgb]{0.78,0.36,0.04}{##1}}}
\expandafter\def\csname PYG@tok@sh\endcsname{\def\PYG@tc##1{\textcolor[rgb]{0.25,0.44,0.63}{##1}}}
\expandafter\def\csname PYG@tok@ow\endcsname{\let\PYG@bf=\textbf\def\PYG@tc##1{\textcolor[rgb]{0.00,0.44,0.13}{##1}}}
\expandafter\def\csname PYG@tok@sx\endcsname{\def\PYG@tc##1{\textcolor[rgb]{0.78,0.36,0.04}{##1}}}
\expandafter\def\csname PYG@tok@bp\endcsname{\def\PYG@tc##1{\textcolor[rgb]{0.00,0.44,0.13}{##1}}}
\expandafter\def\csname PYG@tok@c1\endcsname{\let\PYG@it=\textit\def\PYG@tc##1{\textcolor[rgb]{0.25,0.50,0.56}{##1}}}
\expandafter\def\csname PYG@tok@o\endcsname{\def\PYG@tc##1{\textcolor[rgb]{0.40,0.40,0.40}{##1}}}
\expandafter\def\csname PYG@tok@kc\endcsname{\let\PYG@bf=\textbf\def\PYG@tc##1{\textcolor[rgb]{0.00,0.44,0.13}{##1}}}
\expandafter\def\csname PYG@tok@c\endcsname{\let\PYG@it=\textit\def\PYG@tc##1{\textcolor[rgb]{0.25,0.50,0.56}{##1}}}
\expandafter\def\csname PYG@tok@mf\endcsname{\def\PYG@tc##1{\textcolor[rgb]{0.13,0.50,0.31}{##1}}}
\expandafter\def\csname PYG@tok@err\endcsname{\def\PYG@bc##1{\setlength{\fboxsep}{0pt}\fcolorbox[rgb]{1.00,0.00,0.00}{1,1,1}{\strut ##1}}}
\expandafter\def\csname PYG@tok@mb\endcsname{\def\PYG@tc##1{\textcolor[rgb]{0.13,0.50,0.31}{##1}}}
\expandafter\def\csname PYG@tok@ss\endcsname{\def\PYG@tc##1{\textcolor[rgb]{0.32,0.47,0.09}{##1}}}
\expandafter\def\csname PYG@tok@sr\endcsname{\def\PYG@tc##1{\textcolor[rgb]{0.14,0.33,0.53}{##1}}}
\expandafter\def\csname PYG@tok@mo\endcsname{\def\PYG@tc##1{\textcolor[rgb]{0.13,0.50,0.31}{##1}}}
\expandafter\def\csname PYG@tok@kd\endcsname{\let\PYG@bf=\textbf\def\PYG@tc##1{\textcolor[rgb]{0.00,0.44,0.13}{##1}}}
\expandafter\def\csname PYG@tok@mi\endcsname{\def\PYG@tc##1{\textcolor[rgb]{0.13,0.50,0.31}{##1}}}
\expandafter\def\csname PYG@tok@kn\endcsname{\let\PYG@bf=\textbf\def\PYG@tc##1{\textcolor[rgb]{0.00,0.44,0.13}{##1}}}
\expandafter\def\csname PYG@tok@cpf\endcsname{\let\PYG@it=\textit\def\PYG@tc##1{\textcolor[rgb]{0.25,0.50,0.56}{##1}}}
\expandafter\def\csname PYG@tok@kr\endcsname{\let\PYG@bf=\textbf\def\PYG@tc##1{\textcolor[rgb]{0.00,0.44,0.13}{##1}}}
\expandafter\def\csname PYG@tok@s\endcsname{\def\PYG@tc##1{\textcolor[rgb]{0.25,0.44,0.63}{##1}}}
\expandafter\def\csname PYG@tok@kp\endcsname{\def\PYG@tc##1{\textcolor[rgb]{0.00,0.44,0.13}{##1}}}
\expandafter\def\csname PYG@tok@w\endcsname{\def\PYG@tc##1{\textcolor[rgb]{0.73,0.73,0.73}{##1}}}
\expandafter\def\csname PYG@tok@kt\endcsname{\def\PYG@tc##1{\textcolor[rgb]{0.56,0.13,0.00}{##1}}}
\expandafter\def\csname PYG@tok@sc\endcsname{\def\PYG@tc##1{\textcolor[rgb]{0.25,0.44,0.63}{##1}}}
\expandafter\def\csname PYG@tok@sb\endcsname{\def\PYG@tc##1{\textcolor[rgb]{0.25,0.44,0.63}{##1}}}
\expandafter\def\csname PYG@tok@k\endcsname{\let\PYG@bf=\textbf\def\PYG@tc##1{\textcolor[rgb]{0.00,0.44,0.13}{##1}}}
\expandafter\def\csname PYG@tok@se\endcsname{\let\PYG@bf=\textbf\def\PYG@tc##1{\textcolor[rgb]{0.25,0.44,0.63}{##1}}}
\expandafter\def\csname PYG@tok@sd\endcsname{\let\PYG@it=\textit\def\PYG@tc##1{\textcolor[rgb]{0.25,0.44,0.63}{##1}}}

\def\PYGZbs{\char`\\}
\def\PYGZus{\char`\_}
\def\PYGZob{\char`\{}
\def\PYGZcb{\char`\}}
\def\PYGZca{\char`\^}
\def\PYGZam{\char`\&}
\def\PYGZlt{\char`\<}
\def\PYGZgt{\char`\>}
\def\PYGZsh{\char`\#}
\def\PYGZpc{\char`\%}
\def\PYGZdl{\char`\$}
\def\PYGZhy{\char`\-}
\def\PYGZsq{\char`\'}
\def\PYGZdq{\char`\"}
\def\PYGZti{\char`\~}
% for compatibility with earlier versions
\def\PYGZat{@}
\def\PYGZlb{[}
\def\PYGZrb{]}
\makeatother

\renewcommand\PYGZsq{\textquotesingle}

\begin{document}

\maketitle
\tableofcontents
\phantomsection\label{index::doc}



\part{LE INTRODUÇION}
\label{cap01:introduccionr}\label{cap01:parte-de-introduccion}\label{cap01::doc}\label{cap01:welcome-to-tercera-prueba-sphinx-s-documentacion}

\chapter{UNE PARTÈ}
\label{cap01:primera-seccion}
Çe le tipea un parafe num ¿serà que le titulè se viste bres

Mua priemre formule:
\begin{gather}
\begin{split}X = a_b - C\end{split}\notag
\end{gather}
Une palote

\textbf{Primerè Super intel}

\begin{tabulary}{\linewidth}{|L|L|L|}
\hline
\textsf{\relax 
Prones
} & \textsf{\relax 
Cantitè
} & \textsf{\relax 
Totalitè
}\\
\hline
Primerè pronè
 & 
2
 & 
2
\\
\hline\end{tabulary}



\section{Espero que haya salido una tabla}
\label{cap01:espero-que-haya-salido-una-tabla}
\emph{¿Saldría una tabla?}

\emph{Mi nombre} como ejemplo \includegraphics[width=20pt]{{max}.jpg}: con este tamaño se puede mostrar una imagen en línea con el texto. \emph{Esta es una citación} será????
\begin{quote}

Salió?
\end{quote}

\emph{Sí salió}


\subsection{Por poco}
\label{cap01:elsuelo}\label{cap01:por-poco}
Ya tengo mayores conocimientos sobre el lenguaje RST.
\begin{quote}

Faltan solo algunos \emph{Detalles}
\end{quote}


\subsubsection{Segunda sección}
\label{cap01:segunda}\label{cap01:segunda-seccion}
En esta sección se escribira algo
para mostrar que en un páarafo normal las líneas manuales son omitidas y se hace el que no tiene.
Esta es la tercer línea dentro del rst.


\chapter{PARTE DE MANEJO}
\label{cap01:tercera}\label{cap01:parte-de-manejo}\begin{description}
\item[{Voy a escribir un párrafo y debajo una sangría}] \leavevmode
¿se habrá aplicado la sangría?

\end{description}

No, creo que no.

\begin{Verbatim}[commandchars=\\\{\}]
Ahora es un párrafo que
respeta las líneas manuales
introducidas en el archivo
RST. Debería verse como en
el código.
\end{Verbatim}

Regresa al cap02 donde hay una {\hyperref[cap02:pregunta]{\emph{¿será que puedo volver a este párrafo desde tercera?}}} .


\part{Segunda Sección}
\label{cap02:segunda-seccion}\label{cap02::doc}
Otra tabla de ensayo \footnote[1]{
Se refiere a que estoy aprendiendo a estructurar una tabla con RST.
} de tabla


\chapter{\textbf{Super tabla+}}
\label{cap02:sueldo}\label{cap02:super-tabla}
\begin{tabulary}{\linewidth}{|L|L|L|}
\hline
\textsf{\relax 
Pruebas
} & \textsf{\relax 
Cantidad \protect\footnotemark[2]
} & \textsf{\relax 
Total
}\\
\hline
Segunda prueba
 & 
1
 & 
3
\\
\hline\end{tabulary}

\footnotetext[2]{
La cantidad de veces que voy realizando las pruebas \textbf{y por el momento va yendo nomás} ¿será que mejora \emph{la cosa esta} o tendré que seguir con \emph{LibreOffice} nomás.
}

\section{Espero que haya salido una tabla}
\label{cap02:espero-que-haya-salido-una-tabla}
\emph{¿Saldría una tabla?}

Si salió una tabla ve a la {\hyperref[cap01:segunda]{\emph{Segunda sección}}}

\textbf{No, no salió una tabla}

Mira esto {\hyperref[cap01:tercera]{\emph{PARTE DE MANEJO}}}


\subsection{¿será que puedo volver a este párrafo desde tercera?}
\label{cap02:sera-que-puedo-volver-a-este-parrafo-desde-tercera}\label{cap02:pregunta}
Esto del RST es genial. Cuando quiera un manual multilingüe voy a ensayar a usar esto.


\part{Tercera Sección}
\label{cap03:tercera-seccion}\label{cap03::doc}
Tablas con imágenes Otra tabla

\textbf{Super tabla+}

\begin{tabulary}{\linewidth}{|L|L|L|}
\hline
\textsf{\relax 
Pruebas
} & \textsf{\relax 
Cantidad
} & \textsf{\relax 
Total
}\\
\hline
Segunda prueba
 & 
1
 & 
3
\\
\hline\end{tabulary}



\chapter{Espero que haya salido una tabla}
\label{cap03:espero-que-haya-salido-una-tabla}
\emph{¿Saldría una tabla?}

\textbf{Sí, salió una tabla}

\begin{tabulary}{\linewidth}{|L|L|L|}
\hline
\textsf{\relax 
Meses
} & \textsf{\relax 
Años
} & \textsf{\relax 
Total
}\\
\hline
Tercer prueba
 & 
1
 & 
4
\\
\hline
Con dos columnas
 & 
1
 & 
1
\\
\hline
Una columna
 &  \multicolumn{2}{l|}{
Dos columnas
}\\
\hline\end{tabulary}



\section{Tal vez sea una sub sección}
\label{cap03:tal-vez-sea-una-sub-seccion}\begin{itemize}
\item {} 
Primer elemento de lista

\item {} 
Segundo

\item {} 
Tercero
\begin{itemize}
\item {} 
sub lista?

\item {} 
será?

\end{itemize}

\end{itemize}

\textbf{*Veremos*}


\subsection{Ya ví}
\label{cap03:ya-vi}
¿será que se inserta una imagen aquí?
\begin{quote}

\includegraphics[width=300pt]{{max}.jpg}:
\end{quote}

\emph{si no veo la imagen arriba de esto es que no ess así}

\includegraphics[width=300pt]{{max}.jpg}


\subsubsection{Otra imagen}
\label{cap03:otra-imagen}

\section{¿Una fórmula?}
\label{cap03:una-formula}\begin{gather}
\begin{split}Y = D^t X\end{split}\notag
\end{gather}
where:
\begin{itemize}
\item {} 
\(Y\) = vector of principal components

\item {} 
\(D\) = matrix of eigenvectors of the covariance matrix \(C_x\) in X space

\item {} 
\(t\) denotes vector transpose

\end{itemize}

And \(X\) is calculated as:
\begin{gather}
\begin{split}X = P - M\end{split}\notag
\end{gather}\begin{itemize}
\item {} 
\(P\) = vector of spectral values associated with each pixel

\item {} 
\(M\) = vector of the mean associated with each band

\end{itemize}

The second formula {\hyperref[cap01:introduccionr]{\emph{LE INTRODUÇION}}}
\begin{gather}
\begin{split}Y = D^2 H*5\end{split}\notag
\end{gather}
Explicación:
\begin{quote}

\(D\) = Distancia caminada hoy
\end{quote}

\textgreater{}Todo lo anterior fue copiado del SCP\_Manual

\textbf{Mira:}

Este enlace al documento cap02.rst {\hyperref[cap02:sueldo]{\emph{Super tabla+}}}:

Dime si funciona

Mira {\hyperref[cap01:elsuelo]{\emph{Por poco}}} para mayor información.



\renewcommand{\indexname}{Index}
\printindex
\end{document}
